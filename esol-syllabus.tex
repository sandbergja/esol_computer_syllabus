\documentclass[12pt,article,oneside]{memoir}
\usepackage{comment}
\usepackage{hyperref}
\usepackage{cite}
\usepackage{html}
\author{Jane Sandberg\thanks{You can email me at \htmladdnormallink{sandbej at linnbenton dot edu}{mailto:sandbej@linnbenton.edu}, call me at \htmladdnormallink{(541) 917 4655}	{tel:5419174655}, or stop by my office hours.}}
\title{ESOL 0.792E: Computer Basics for Non-native English Speakers\thanks{Mondays and Wednesdays 11am-12:20pm in WH-222}}
\begin{document}
\renewcommand{\labelitemi}{$\triangleright$}
\setcounter{secnumdepth}{0}
\tightlists


\maketitle

\begin{htmlonly}
\tableofcontents
\end{htmlonly}
\section{Office hours}

\subsection{Mondays}
10:30am-12pm in WH-143 (my office in the LBCC Library)

\section{Course Description}
This course will provide an overview of computer basics for non­native speakers of English. The course content will cover various aspects of computer usage at the basic level including: knowing parts of the computer, how to turn on and shut­down a computer, how to use a mouse, how to navigate the Internet, keyboarding, creating a document in Microsoft Word, etc.

\section{Course Learning Outcomes}
By the end of this course, you will be able to:
\begin{itemize}
 \item Name and use the basic parts of a computer
 \item Improve typing skills 
 \item Check and send email,  including email with attachments
 \item Use a tabbed web browser
 \item Download, modify, save, and locate files 
 \item Fill out HTML forms
 \item Navigate through and between web pages
\end{itemize}

\section{Computers on our campus}

Anybody may use the computers at the LBCC Library. You may also borrow laptops from the Library.  If you have questions about your LBCC computer accounts, you can ask for help at the Student Help Desk, which is located in the Library.  The Library is open Monday-Thursday: 7:30am-9pm, Friday: 7:30am-5pm, Saturday: 10am-3pm, Sunday: 1-6pm.

\section{Accessibility Resources}

The Center for Accessibility Resources provides reasonable accommodations, academic adjustments and auxiliary aids to ensure that qualified students with disabilities have access to classes, programs and events at LBCC. Students are responsible for requesting accommodations in a timely manner. To receive appropriate and timely accommodations from LBCC, please give the Center for Accessibility Resources as much advance notice of your disability and specific needs as possible, as certain accommodations such as  sign language interpreting take days to weeks to have in place. Contact the Disability Coordinator at LBCC; RCH - 105l; 6500 Pacific Blvd. SW Albany, OR 97330. Phone: 541-917-4789 or via OR Telecommunications Relay TTD at 1-800-735-2900 or 1-800-735-1232.

\section{Respect}

The LBCC community is enriched by diversity. Everyone has the right to think, learn, and work together in an environment of respect, tolerance, and goodwill. I actively support this right regardless of race, creed, color, personal opinion, gender, sexual orientation, or any of the countless other ways in which we are diverse.  

% http://media.gcflearnfree.org/assets/interactives/typing/index.html

\section{Credentials}
\begin{itemize}
\item Email address: \rule{4in}{.1pt} \\
\vspace{1in}
\item Email password: \rule{4in}{.1pt} \\
\end{itemize}



\end{document}
